\documentclass[12pt]{article}



\usepackage{graphicx}
\graphicspath{{images/}}

\title{my}


\begin{document}


\centering\huge\textbf{Assignment}

\centering\large on

\centering\huge\textbf{Any Five Medical Device}


\vspace{1cm}

\centering\LARGE\textit{submitted by:}

\centering\LARGE\textbf{Itisha Kaiwart}

\begin{center}
\begin{Large}
\textbf{Roll No:21111024}\\
\vspace{0.4cm}
\textbf{1st semester,Biomedical Engineering}
\end{Large}
\end{center}


\vspace{0.5cm}

\centering\Large\textbf{National Institute Of Technology}

\begin{figure}[h]
\centering
\includegraphics[scale=1]{nitrr.jpg}
\end{figure}

\centering\LARGE\textit{under the supervision of: }      \\
 \centering\Large\textbf{Dr. Saurabh Gupta}

\pagebreak

\tableofcontents

\pagebreak 
\section{\LARGE\centering\textbf{ACKNOWLEDGEMENT}}
 
\large\flushleft The success and final outcome of this assignment required a lot of guidance and assistance from many people and I extremely fortunate to have got this all along the completion of my assignment work. Whatever I have done is only due to such guidance and assistance and we would not forget to thank them. I respect and thank Dr.Saurabh Gupta for giving we an opportunity to do this assignment work and providing us all support and guidance which made me complete the assignment on time, We extremely grateful to him for providing such a nice support and guidance.

\vspace{1cm}
I am really grateful because I managed to complete this assignment within the time given by Dr. Saurabh Gupta. 
\pagebreak

\section{\LARGE\centering\textbf{INTRODUCTION}}

\large\flushleft Biomedical/medical devices are instruments, machines, implants, in vitro reagents, software, materials, or other related articles that are purposed for the safe and effective prevention, diagnosis, treatment, and rehabilitation of illness and disease for human beings.\\ 

\vspace{1cm}
Ther are thousands of medical device . I am reporting about 5 medical devices are:
\begin{itemize}
\item Electrocardiography(ECG)
\item Capsule Endoscopy
\item Robot-Assisted Surgery
\item Portable Ultrasound Device
\item Cardiopulmonary Bypass (CPB)
\end{itemize}
\pagebreak
 
\section{\centering\huge\textbf{Electrocardiography(ECG)}}


\vspace{0.5cm}

\LARGE\flushleft\textbf{Introduction}\\
\large\flushleft An electrocardiogram (ECG) is a simple test that can be used to check your heart's rhythm and electrical activity.

Sensors attached to the skin are used to detect the electrical signals produced by your heart each time it beats.

These signals are recorded by a machine and are looked at by a doctor to see if they're unusual.

An ECG may be requested by a heart specialist (cardiologist) or any doctor who thinks you might have a problem with your heart, including your GP.

The test can be carried out by a specially trained healthcare professional at a hospital, a clinic or at your GP surgery.

Despite having a similar name, an ECG isn't the same as an echocardiogram, which is a scan of the heart.


\LARGE\flushleft\textbf{When an ECG is used ?}

\large\flushleft An ECG is often used alongside other tests to help diagnose and monitor conditions affecting the heart.

It can be used to investigate symptoms of a possible heart problem, such as chest pain, palpitations (suddenly noticeable heartbeats), dizziness and shortness of breath.\\

An ECG can help detect:\\

\begin{enumerate}
\item arrhythmias – where the heart beats too slowly, too quickly, or irregularly.\\
\item coronary heart disease – where the heart's blood supply is blocked or interrupted by a build-up of fatty substances.\\
\item heart attacks – where the supply of blood to the heart is suddenly blocked\\
\item cardiomyopathy – where the heart walls become thickened or enlarged.\\
A series of ECGs can also be taken over time to monitor a person already diagnosed with a heart condition or taking medication known to potentially affect the heart.

\end{enumerate}

\vspace{1cm}


\begin{figure}[h]
\centering
\includegraphics[scale=1]{E}
\end{figure}

\pagebreak

\LARGE\flushleft\textbf{Types of ECG}

\large\flushleft There are 3 main types of ECG:\\

\begin{enumerate}
\item A resting ECG – carried out while you're lying down in a comfortable position\\
\item A stress or exercise ECG – carried out while you're using an exercise bike or treadmill.\\
\item An ambulatory ECG (sometimes called a Holter monitor) – the electrodes are connected to a small portable machine worn at your waist so your heart can be monitored at home for 1 or more days.\\

\end{enumerate}
The type of ECG you have will depend on your symptoms and the heart problem suspected.\\

For example, an exercise ECG may be recommended if your symptoms are triggered by physical activity, whereas an ambulatory ECG may be more suitable if your symptoms are unpredictable and occur in random, short episodes.




\LARGE\flushleft\textbf{Risks or side effects}

\large\flushleft An ECG is a quick, safe and painless test. No electricity is put into your body while it's carried out.

There may be some slight discomfort when the electrodes are removed from your skin – similar to removing a sticking plaster – and some people may develop a mild rash where the electrodes were attached.


\vspace{1.5cm}

\centering\huge-------***------




\pagebreak


\section{\centering\Huge\textbf{Capsule Endoscopy}}


\vspace{0.5cm}


\LARGE\flushleft\textbf{Introduction}

\vspace{0.2cm}
\Large\flushleft {Capsule endoscopy is a procedure that uses a tiny wireless camera to take pictures of your digestive tract. A capsule endoscopy camera sits inside a vitamin-size capsule you swallow. As the capsule travels through your digestive tract, the camera takes thousands of pictures that are transmitted to a recorder you wear on a belt around your wear on a belt around your waist.}

\begin{figure}[h]
\centering
\includegraphics[scale=1]{camerapill}
\end{figure}
\LARGE\flushleft\textbf{Why it's done ?}

\large\flushleft Your doctor might recommend a capsule endoscopy procedure to:

\begin{itemize}


\item Find the cause of gastrointestinal bleeding. The most common reason for doing capsule endoscopy is to explore unexplained bleeding in the small intestine.
Diagnose inflammatory bowel diseases, such as Crohn's disease. Capsule endoscopy can reveal areas of inflammation in the small intestine.\\
\item Diagnose cancer. Capsule endoscopy can show tumors in the small intestine or other parts of the digestive tract.\\
\item Diagnose celiac disease. Capsule endoscopy is sometimes used in diagnosing and monitoring this immune reaction to eating gluten.\\
\item Examine your esophagus. Capsule endoscopy has also been approved to evaluate the muscular tube that connects your mouth and your stomach (esophagus) to look for abnormal, enlarged veins (varices).\\
\item Screen for polyps. People who have inherited syndromes that can cause polyps in the small intestine might occasionally undergo capsule endoscopy.\\

\end{itemize}
Do follow-up testing after X-rays or other imaging tests. If the results of an imaging test are unclear or inconclusive, your doctor might recommend a capsule endoscopy to get more information.


\LARGE\flushleft\textbf{Risks}

\large\flushleft Capsule endoscopy is a safe procedure that carries few risks. However, it's possible for a capsule to become lodged in your digestive tract rather than leaving your body in a bowel movement within several days.

The risk, which is small, might be higher in people who have a condition — such as a tumor, Crohn's disease or previous surgery in the area — that causes a narrowing (stricture) in the digestive tract. If you have abdominal pain or are at risk of a narrowing of your intestine, your doctor likely will have you get a CT scan to look for a narrowing before using capsule endoscopy. Even if the CT scan shows no narrowing, there's still a small chance that the capsule could get stuck.

If the capsule hasn't passed in a bowel movement but isn't causing signs and symptoms, your doctor might give the capsule more time to leave your body. However, a capsule causing signs and symptoms that indicate bowel obstruction must be removed, either by surgery or through a traditional endoscopy procedure, depending on where the capsule is stuck.


\vspace{1.5cm}

\centering\huge-------***------



\pagebreak
\section{\centering\Huge\textbf{Robbot-Assisted Surgery}}

\vspace{0.5cm}

\LARGE\flushleft\textbf{Introduction}

\large\flushleft Robotic surgery, or robot-assisted surgery, allows doctors to perform many types of complex procedures with more precision, flexibility and control than is possible with conventional techniques. Robotic surgery is usually associated with minimally invasive surgery — procedures performed through tiny incisions. It is also sometimes used in certain traditional open surgical procedures.

\vspace{1cm}

\begin{figure}[h]
\centering
\includegraphics[scale=0.5]{Robot.jpg}
\end{figure}

\pagebreak

\LARGE\flushleft\textbf{Advantages}

\large\flushleft Surgeons who use the robotic system find that for many procedures it enhances precision, flexibility and control during the operation and allows them to better see the site, compared with traditional techniques. Using robotic surgery, surgeons can perform delicate and complex procedures that may have been difficult or impossible with other methods.\\

\vspace{0.4cm}
Often, robotic surgery makes minimally invasive surgery possible. The benefits of minimally invasive surgery include:\\
\begin{itemize}


\item Fewer complications, such as surgical site infection
Less pain and blood loss\\
\item Quicker recovery\\
\item Smaller, less noticeable scars.\\

\end{itemize}

\LARGE\flushleft\textbf{Risks}

\large\flushleft Robotic surgery involves risk, some of which may be similar to those of conventional open surgery, such as a small risk of infection and other complications.\\
\vspace{1.5cm}

\centering\huge-------***------
\pagebreak
\section{\centering\Huge\textbf{Portable Ultrasound Device}}

\LARGE\flushleft\textbf{Introduction}

\large\flushleft Portable ultrasound is a modality of medical ultrasonography that utilizes small and light devices, compared to the console-style ultrasound machines that preceded them. In most cases these mobile ultrasound systems could be carried by hand and in some cases even operated for a time on battery power alone. The first portable ultrasound machines arrived in the early 1980s but battery powered systems that could be easily carried did not arrive until the late 1990s.

\vspace{1cm}

\begin{figure}[h]
\centering
\includegraphics[scale=1]{pus}
\end{figure}

 \pagebreak
\LARGE\flushleft\textbf{Uses of ultrasound test}

\large\flushleft Ultrasound imaging has many uses in medicine, from confirming and dating a pregnancy to diagnosing certain conditions and guiding doctors through precise medical procedures.
\begin{itemize}


\item Pregnancy. Ultrasound images have many uses during pregnancy. Early on, they may be used to determine due dates, reveal the presence of twins or other multiples, and rule out ectopic pregnancies. They also are valuable screening tools in helping to detect potential problems, including some birth defects, placental issues, breech positioning, and others. Many expectant parents look forward to learning the sex of their babies via ultrasound midway through a pregnancy. And later in pregnancy, doctors can even use ultrasounds to estimate how large a baby is just before delivery.

\item Diagnostics. Doctors employ ultrasound imaging in diagnosing a wide variety of conditions affecting the organs and soft tissues of the body, including the heart and blood vessels, liver, gallbladder, spleen, pancreas, kidneys, bladder, uterus, ovaries, eyes, thyroid, and testicles. Ultrasounds do have some diagnostic limitations, however; sound waves do not transmit well through dense bone or parts of the body that may hold air or gas, such as the bowel.


\item Use during medical procedures. Ultrasound imaging can help doctors during procedures such as needle biopsies, which require the doctor to remove tissue from a very precise area inside the body for testing in a lab.

\item Therapeutic applications. Ultrasounds sometimes are used to detect and treat soft-tissue injuries.


\end{itemize}

\LARGE\flushleft\textbf{Advantages of ultrasound}

\large\flushleft Ultrasounds offer many advantages:
\begin{itemize}

\item They are generally painless and do not require needles, injections, or incisions.
\item Patients aren't exposed to ionizing radiation, making the procedure safer than diagnostic techniques such as X-rays and CT scans. In fact, there are no known harmful effects when used as directed by your health care provider.
\item Ultrasound captures images of soft tissues that don't show up well on X-rays.
\item Ultrasounds are widely accessible and less expensive than other methods.
\end{itemize}

\vspace{1cm}
\centering\huge ------***-----
\pagebreak


\section{\centering\Huge\textbf{Cardiopulmonary Bypass}}

\LARGE\flushleft\textbf{Introduction}

\large\flushleft A cardiopulmonary bypass machine (CBM) is commonly known as a heart-lung bypass machine. It is a device that does the work of providing blood (and oxygen) to the body when the heart is stopped for a surgical procedure.\\


In most cases, the machine is used to perform serious procedures that require the heart to be stopped. Patients are on the pump only as long as it takes to stop the heart from beating, complete open-heart surgery or a procedure on the lungs, and restart the heart.

\vspace{1cm}
\begin{figure}[h]
\centering
\includegraphics[scale=1]{CPB.jpg}
\end{figure}

\pagebreak

\LARGE\flushleft\textbf{How does cardiopulmonary bypass work}

\large\flushleft The surgeon attaches special tubing to a large blood vessel (like starting a very large IV) that allows oxygen-depleted blood to leave the body and travel to the bypass machine. There, the machine oxygenates the blood and returns it to the body through the second set of tubing, also attached to the body.3 The constant pumping of the machine pushes the oxygenated blood through the body, much like the heart does.\\

The placement of the tubes is determined by the preference of the surgeon. The tubes must be placed away from the surgical site so they do not interfere with the surgeon’s work, but placed in a blood vessel large enough to accommodate the tubing and the pressure of the pump. The two tubes ensure that blood leaves the body before reaching the heart and returns to the body after the heart, giving the surgeon a still and mostly bloodless area to work.\\


A third tube is also inserted very near or directly into the heart, but not connected to the CPM. It is used to flush the heart with cardioplegia, a potassium solution which stops the heart.7 Once the cardioplegia takes effect, the CBM is initiated and takes over the heart and lung function.\\

\LARGE\flushleft\textbf{Need}

\large\flushleft Coronary bypass surgery is one treatment option if you have a blocked artery to your heart. You and your doctor might consider it if: You have severe chest pain caused by narrowing of several arteries that supply your heart muscle, leaving the muscle short of blood during even light exercise or at rest.

\LARGE\flushleft\textbf{Risks}

\large\flushleft The risks of being on heart and lung bypass include blood clots, bleeding after surgery, surgical injury to the phrenic nerve, acute kidney injury, and decreased lung and/or heart function. These risks are decreased with shorter times on the pump and increased with longer pump times. 



\vspace{1.5cm}

\centering\huge-------***------
\pagebreak

\section{\LARGE\centering references}
\large\flushleft wikipedia,mayoclinic,www.nhs.uk
\end{document}